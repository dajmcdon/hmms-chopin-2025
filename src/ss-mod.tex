\documentclass[crop,tikz,convert=pdf2svg]{standalone}
% \usetikzlibrary{...}% tikz package already loaded by 'tikz' option
\definecolor{lgreen}{HTML}{00AF64}
\definecolor{dblue}{HTML}{084b7f}
\definecolor{ddblue}{HTML}{053b64}
\definecolor{lred}{HTML}{FF9B73}
\definecolor{nred}{HTML}{FF4900}
\definecolor{dred}{HTML}{C63800}
\definecolor{norange}{HTML}{FF9200}
\definecolor{dorange}{HTML}{C67100}
\definecolor{nyellow}{RGB}{255,221,0}
\definecolor{ngreen}{HTML}{00AF64}
\definecolor{dgreen}{HTML}{00884e}
\definecolor{nblue}{HTML}{0B61A4}
\definecolor{jblue}{HTML}{0B61A4}
\usetikzlibrary{fit,arrows.meta,automata}					
\usetikzlibrary{backgrounds}


\begin{document}
\tikzstyle{state}=[circle, line width=2pt,minimum size=1.25cm, draw=nblue]
\tikzstyle{measurement}=[circle,thick,minimum size=1.25cm,draw=norange,fill=norange]
\tikzstyle{switch}=[rectangle,line width=2pt, minimum size=1cm, draw=nblue]
\begin{tikzpicture}[text height=1.5ex,text depth=0.25ex,ampersand replacement=\&]
  % "text height" and "text depth" are required to vertically
  % align the labels with and without indices.
  
  % The various elements are conveniently placed using a matrix:
  \matrix[row sep=1cm,column sep=1cm] {
    % Second line: hidden continuous state
    \node (x_k-2) {$\cdots$}; \&
    \node (x_k-1) [state] {$\mathbf{x}_{k-1}$}; \&
    \node (x_k)   [state] {$\mathbf{x}_k$};     \&
    \node (x_k+1) [state] {$\mathbf{x}_{k+1}$}; \&
    \node (x_k+2) {$\cdots$}; \\
    % Third line: Measurement
    \node (y_k-2) {$\cdots$}; \&        
    \node (y_k-1) [measurement] {$\mathbf{y}_{k-1}$}; \&
    \node (y_k)   [measurement] {$\mathbf{y}_k$};     \&
    \node (y_k+1) [measurement] {$\mathbf{y}_{k+1}$}; \&
    \node (y_k+2) {$\cdots$};\\
  };
  % The diagram elements are now connected through arrows:
  \path[>=Stealth, scale=10, line width=2pt,->]
  (x_k-2) edge (x_k-1)
  (x_k-1) edge (x_k)	
  (x_k)   edge (x_k+1)	
  (x_k+1)   edge (x_k+2)	
  
  (x_k-1) edge (y_k-1)
  (x_k) edge (y_k)
  (x_k+1)   edge (y_k+1);
  % (s_k-1) edge[bend left=40] (y_k-1)
  % (s_k) edge[bend left=40] (y_k)
  % (s_k+1)   edge[bend left=40] (y_k+1);
\end{tikzpicture}
\end{document}